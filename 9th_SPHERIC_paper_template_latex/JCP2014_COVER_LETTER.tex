\documentclass[a4paper,12pt]{article}
\usepackage{graphicx}
%%%% \usepackage{fancyhdr}
\usepackage{mathrsfs}
%\usepackage{psfig}
\usepackage{epsfig}
\usepackage{times}
\usepackage{amsmath}
\usepackage{amssymb}
\usepackage[verbose]{geometry}
%\usepackage{rotating}
\usepackage{multirow}
\usepackage{color}
\usepackage{hyperref}
\usepackage[latin1]{inputenc}
\usepackage{verbatim}
\usepackage{natbib}
%%%%%%%%%%%%%%%%%%%%%%%%%%%%%%%%%%%%%%%%%%%%%%%%%%%%%%%%%%%%%%%%%%%%%%%%%%%%%%%%%%%%%%%%%%%
\newcommand{\rc}{\\[0.1 in]}
%%%%%%%%%%%%%%%%%%%%%%%%%%%%%%%%%%%%%%%%%%%%%%%%%%%%%%%%%%%%%%%%%%%%%%%%%%%%%%%%%%%%%%%%%%%
\geometry{top=2.5cm, bottom=4.5cm, left=2.5cm, right=2.5cm}
\setlength{\topskip}{1.0cm} \setlength{\headsep}{2.0cm}
\setlength{\footskip}{1.5cm}
\parskip=0pt
\parindent=0pt
%%%%\setlength{\textwidth}{16.5cm}
%%%%\setlength{\textheight}{23.0cm}
%%%%\setlength{\oddsidemargin}{-.40cm}

\renewcommand{\familydefault}{cmss}
\renewcommand{\seriesdefault}{m}
\DeclareSymbolFontAlphabet{\mathcal}{rsfs}
\bibliographystyle{elsart-harv}
%%%%%%%%%%%%%%%%%%%%%%%%%%%%%%%%%%%%%%%%%%%%%%%%%%%%%%%%%%%%%%%%%%%%%%%%%%%%%%%%%%%%%%%%%%%
%%%%%%%%%%%%%%%%%%%%%%%%%%%%%%%%%%%%%%%%%%%%%%%%%%%%%%%%%%%%%%%%%%%%%%%%%
\begin{document}
\thispagestyle{empty}
%
\noindent Dr. George E. Karniadakis, Assoc. Editor, JCP  

Brown University, 

Providence, Rhode Island, USA$$$$


\hspace{6cm}Juan M. Gimenez.

\hspace{6cm}Centro de Investigaci\'on de M\'etodos Computacionales (CIMEC).

\hspace{6cm}Universidad Nacional del Litoral /CONICET.

\hspace{6cm}email: jmarcelogimenez@gmail.com

\hspace{6cm}http://www.cimec.org.ar

\hspace{6cm}

\hspace{6cm}Leo M. Gonz\'{a}lez

\hspace{6cm}Universidad Polit\'{e}cnica de Madrid (ETSIN-UPM)

\hspace{6cm}email: leo.gonzalez@upm.es

\hspace{6cm}http://www.cfm.upm.es/

\vspace{2cm}

$\quad$\hspace{10cm}Madrid, May. 13th, 2014
$$$$

\noindent Dear Sir:\rc
%
Please find enclosed for your kind review the manuscript entitled:
%
\begin{center}
``An efficient adaptation of the particle finite element method to free surface flows''
\end{center}
%
In this paper, a new generation of the particle method known as Particle Finite
 Element Method (PFEM), which combines convective particle movement and a fixed 
 mesh resolution, is applied to free surface flows. Two novel steps are introduced: first, 
 the possibility of using larger time steps, which shows that shorter computational times 
 can be achieved while maintaining the accuracy of the solution in a wide variety of problems. 
 Second, since surface flows are the main topic of this paper, different improved versions of 
 discontinuous and continuous enriched basis functions for the pressure field have also been developed. 
 Combining these different improvements, a wide variety of free surface flows have been solved in 2D and 3D cases.

%
To our knowledge these topics have not been covered in other publications.
%
%
Thank you for your time and please report any mistake that we could be
committing in this submission.
%
We hope that the editorial committee will find the paper interesting and worth
publishing.

%
$$$$
Sincerely,
$$$$
%$\quad$$\quad$$\quad$Juan M. Gimenez
$\quad$$\quad$$\quad$The Authors
$$\quad$$
%\bibliography{./../../../BIBTEX/bib}
% \bibliography{bib}
\end{document}
