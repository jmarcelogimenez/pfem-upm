An accurate and efficient simulation of the interface evolution is crucial in the simulation of free-surface flows. During the flow evolution, it is essential that the interface remains sharp. Large jumps of fluid density and viscosity across the interface should be correctly assumed by the numerical algorithm in order to satisfy the momentum balance at the vicinity of the interface.

Methods used to describe the evolution of interfaces can be clustered in two classes, namely: interface capturing and interface tracking methods. While in the former the interface is determined by an implicit function that is advected in a Eulerian frame (see Volume of Fluid \cite{VoF} and Level Set\cite{Osher01}), in the latter the interface evolution equation is solved in a Lagrangian fashion, for example, by evolving marker particles. The spatial domain discretization using mesh and particles allows PFEM-2 to select an appropriate combination of those approaches where the free-surface position information is shared and interchanged by the particles and the fixed mesh.

There are several, albeit small, differences between the PFEM-2 algorithm for homogeneous flows and the two fluid version. Those differences stem from the density and viscosity discontinuities that appear in the fluid, consequently most of the implemented changes are related to the strategies followed to correctly capture the interface between both fluids. Although some details presented in this section have already been reported in \cite{Idelsohn13c}, important strategies that significantly improve the accuracy and efficiency of the computation were added in this work.

Taking into account the Algorithm presented in section \ref{PFEM_Algorithm}, important considerations to manage and improve each one of the stages for the particular case of multiphases problems are presented hereafter. The following part is related with three aspects of the simulation, namely: the kinematic treatment of the fluid particles during the X-IVAS stage, the enrichment technique for the free-surface definition and the pressure computation step. Although these three topics are listed independently, the are closely related to each other during the computation and, consequently, will be treated together in the next section.

\subsection{Internal interfaces tracking}\label{sec:tracking}

When two different fluids separated by an interface are considered, each particle $p$ carries the information of the fluid to which it was initially assigned. This quantity, represented by a scalar function $\lambda_p$, has integer values $-1$ or $1$ depending if it belongs to the first or second fluid. This value is advected, adding one equation to the \textit{Streamline Integration Stage}: $\frac{D\lambda}{Dt}=0$, i.e. each particle keeps its marker value during the entire simulation. This function is projected to the mesh nodes to determine the free-surface position. Mesh nodes thus obtain real values after the projection which are different to the integer values $\pm1$ that the particles transport. The free-surface interface is defined as the set of points that satisfy the equation $\lambda=0$.

In general, the movement of a particle is done by sub-steps according to equation (\ref{Step2astep}). The velocity used in the particle movement at position $\mathbf{x}_p$ is calculated by the equation:

\begin{equation}\label{Interpolation}
    \displaystyle \mathbf{v}(\mathbf{x}_p)=\frac{\displaystyle \sum_{i}\mathbf{v}_i^n\psi_i(\mathbf{x}_p)}{\displaystyle \sum_{i}\psi_i}
\end{equation}

where the nodes included in the interpolation are the nodes of the hosting element. Two situations could happen to any particle when its position changes: all the nodes of the hosting element have the same density as the fluid particle or one or more nodes have a different density than the fluid particle. Whilst in the first case a typical finite element interpolation is performed, the second situation clearly represents a case where the fluid particle is close to the interface. In the particular case that the density ratio $\rho_1/\rho_2$ is larger than a first numerical parameter $\alpha$, that is $\rho_1/\rho_2>\alpha$, two situations can appear:

 \begin{itemize}
    \item $\rho_p=\rho_2$ (light particle tracking). The velocity will be computed using Equation (\ref{Interpolation}).
    \item $\rho_p=\rho_1$ (heavy particle tracking). Depending on the value of $A=\sum_{i(\rho_i=\rho_p)}\psi_i$ where the sum is limited to the hosting nodes that have the same density as the particle, we can have 2 possibilities:
    \begin{itemize}
      \item $A<\beta$  the gravity force will be included in the computation of the particle trajectory, which will finally be computed as a parabolic motion.
      \item $A>\beta$, the sums that appear in Equation (\ref{Interpolation}) are both restricted to the hosting nodes $i$ that have the same density as the particle $\rho_i=\rho_p$.
    \end{itemize}
    being $\beta$ a second numerical parameter which indicates the minimum amount of information needed to move a heavy particle following the streamlines of its own phase. For those cases where the density ratio $\rho_1/\rho_2<\alpha$, equation (\ref{Interpolation}) is always used.
 \end{itemize}

This means that if a water particle is momentarily in an air regime, it will remain as a water particle for further determination of the interface position. While a better approximation to the real particle trajectory defined by the acting forces is searched, the parabolic motion is used herein as the simplest trajectory when only gravity forces are acting, an interesting alternative for the particle motion (not used in this work) could be using a water droplet drag model.

Once the particles have been advected and the intermediate velocities have been determined, this updated velocity and density information has to be incorporated into the mesh nodes. During this \textit{Projection Stage}, each mesh node $j$ updates its intermediate velocity according to equation (\ref{Step3a}), analogously a similar equation is used to update the values of $\lambda_j$. Depending on the value of $\lambda_j$ the instantaneous local interface inside each element is determined as the iso-line (an iso-plane in 3D) where $\lambda(\xx)=0$.

\subsection[Enriched Shape Functions]{Shape function enrichments for pressure gradient discontinuity capturing}

In typical finite element methods, gradient of the shape functions $\nabla\phi_j$ are continuous within each element, and therefore any interpolated unknown is also continuous. When the interface crosses an element, the discontinuity in the material properties leads to discontinuities in the unknows and/or its gradients that classical interpolations do not capture. These are:
\begin{itemize}
\item pressure gradient gaps where density discontinuities are present\cite{Coppola05}
\item pressure gaps where viscosity discontinuities are present\cite{Idelsohn10}
\item gradient velocity gaps where viscosity discontinuities are present
\item pressure gaps where surface tension is present
\end{itemize}
For the case of interest presented in this work where two different density fluids are simulated, the interpolation errors in the pressure and its gradient give rise to spurious velocities that can render the solution meaningless.

Enrichment methods add degrees of freedom to elements that are cut by the interface in order to reduce interpolation errors. In this work, two space enrichment methodologies are proposed in order to treat pressure gradient discontinuities.

The first enriched space is based on the one presented by Coppola\cite{Coppola05}, which is illustrated, for the two-dimensional case, in Figure \ref{fg:enrichment1}. The triangle conformed by the nodes at positions $\xx_1$, $\xx_2$ and $\xx_3$ is cut by the interface at points $\xx_A$ and $\xx_B$, dividing the element into two regions $\Omega_1$ and $\Omega_2$. The construction of the enrichment function $\phi^*$ must satisfy

\begin{equation}
   \left\{\;
   \begin{matrix}
      \phi^*(\xx_A)= 1 \\
      \phi^*(\xx_1)= \phi^*(\xx_2)=\phi^*(\xx_3)= 0
   \end{matrix}\;
   \right.
   \label{eq:enrich-1a}
\end{equation}
where $\xx_A$ is a point over the edge $\xx_1-\xx_2$ such as $\lambda(\xx_A)=0$.

It can be demonstrated \cite{Coppola05} that the enrichment function, which accomplishes the previously presented restrictions, can be expressed as a linear combination of the traditional shape functions, being:
 \begin{align}
    \phi^*|_{\Omega_2} = & \ k_1 \phi_1 \label{phi_enrichment-2}\\
    \phi^*|_{\Omega_1} = & \ k_2 \phi_2 + k_3 \phi_3 \label{phi_enrichment-1}
  \end{align}
where $k_1 = \dfrac{\lambda_2-\lambda_1}{\lambda_2}$, $k_2 = \dfrac{\lambda_1-\lambda_2}{\lambda_1}$ and $k_3 = -k_1\dfrac{\lambda_3}{\lambda_1}$, being $\lambda_j$ the value of the marker function at node $j$. The unique new degree of freedom could be statically condensed within each element in the pressure equation and then recovered in the correction step.

\begin{figure}[H]
  \centering
  \includegraphics[width=.98\columnwidth]{images/enrichment1_r1.pdf}
   \caption{2D interface element. The interface is calculated cutting the element at the segment $A-B$. The enrichment proposed by Coppola (left) and the partition of the triangle into three sub-triangles with its own Gauss points (right).}
   \label{fg:enrichment1}                %% Etiqueta para la figura entera
\end{figure}

The second set of enrichment functions used in this work is described in Figure \ref{fg:enrichment2}. As in the previous case, the two new degrees of freedom can also be statically condensed. However, using this new space, it is possible to ensure continuity between elements at the cost of having to rebuild the system matrix at each time step, which could be an expensive task due to memory allocation.
This enrichment space is constructed following
\begin{equation}
   \left\{\;
   \begin{matrix}
     \phi_A^*(\xx_A)=\phi_B^*(\xx_B)=1 \\
     \phi_A^*(\xx_1)=\phi_A^*(\xx_2)=\phi_A^*(\xx_3)=\phi_A^*(\xx_B)=0 \\
     \phi_B^*(\xx_1)=\phi_B^*(\xx_2)=\phi_B^*(\xx_3)=\phi_B^*(\xx_A)=0
   \end{matrix}\;
   \right.
   \label{eq:enrich-2a}
\end{equation}

\begin{figure}[H]
  \centering
   \includegraphics[width=.9\columnwidth]{images/enrichment2_r1.pdf}
   \caption{2D interface element. The interface is calculated cutting the element at the segment $A-B$. An enrichment space with two functions per interface element, which can be used to ensure continuity between elements, is presented. The integration partition is the same as presented above (Figure \ref{fg:enrichment1}).}
   \label{fg:enrichment2}
\end{figure}

% \begin{figure}[H]
%   \centering
%     \subfloat[]{
% 	  \label{fg:enrichment1}         %% Etiqueta para la primera subfigura
% 	  \includegraphics[width=.9\columnwidth]{images/enrichment1.pdf}
%     } \\
%     %%----segunda subfigura----
%     \subfloat[]{
% 	  \label{fg:enrichment2}         %% Etiqueta para la segunda subfigura
% 	  \includegraphics[width=.9\columnwidth]{images/enrichment2.pdf}
%     }
%    \caption{2D interface element. The interface is calculated cutting the element with the segment $A-B$. Figure \ref{fg:enrichment1} shows the enrichment proposed by Coppola and the partition of the triangle into three sub-triangles with its own Gauss points to allowing to calculate an enhanced integration. Figure \ref{fg:enrichment2} presents an enrichment space with two functions per interface element, which can be used to ensure continuity between elements. The integration partition is the same as presented above.}
%    \label{fg:enrichment}                %% Etiqueta para la figura entera
% \end{figure}

Therefore, using any of the enriched spaces, the pressure is now interpolated in the cut element following:
 \begin{equation}
      p_h(\xx) = \sum_{i=1}^{N_n} \phi_i(\xx) \ p_i + \sum_{i=1}^{N_e} \phi_i^*(\xx) \ p_i^*
   \end{equation}
where $\phi_i$ are traditional linear shape functions (a total of $N_n$), and $\phi_i^*$ are the enrichment shape functions (a total of $N_e$).

In order to capture the discontinuities and taking advantage of the enrichment functions used, the integration rules need to be modified in elements cut by the free surface. The method used is to divide each tetrahedra (triangles when represented in 2D) element into up to six tetrahedral
(three triangular in 2D) sub elements. For each sub element, the same integration rule as for the non-cut elements is used. Figure \ref{fg:enrichment1} shows such a partition where the small circles represent the Gauss points for the integration. When using enrichment functions for the pressure, the material properties $\rho$,$\mu$ are taken as $\rho_1$,$\mu_1$ or $\rho_2$,$\mu_2$, depending on which part of the domain ($\Omega_1$ or $\Omega_2$) the integration point is found.

   \subsection{Pressure Calculation}

Particles can move across several elements and cross the interface during a time-step, consequently, the pressure gradient of the previous time-step would introduce a poor and even unstable approximation of the new pressure forces. In order to avoid these large errors in the evaluation of the pressure gradients in the initial value of the iterative process, the value of the pressure is set to zero at the beginning of each time-step. Therefore, the acceleration over the particle calculated in X-IVAS stage is only due to the gravitational force (the pressure gradient is considered null and viscous forces are treated implicitly)\cite{Idelsohn13c}.

Starting from Equation (\ref{Step5a}), and following the classical variational formulation for the global domain $\Omega$ with boundary $\Gamma$, through integration weighting with test functions and weakening both pressure laplacian and velocity divergence terms, it is possible to obtain the system:

\begin{equation}
   \left[\Delta t \int_{\Omega} \frac{1}{\rho} \nabla \phi^T \nabla \phi \ d\Omega\right]\ \delta p^{n+1} = \left[\int_{\Omega} \nabla \phi^T \psi \ d\Omega\right]\ \hat \vv_j^{n+1}
\label{poisson}
\end{equation}
where the term resulting from weakening
\begin{equation}
\displaystyle \int_{\Gamma} \phi \left[ \hat \vv_j^{n+1} + \Delta t \frac{1}{\rho} \nabla  \delta p^{n+1} \right] \cdot \mathbf{n} \ d\Gamma = \int_{\Gamma} \phi \ \vv_j^{n+1} \cdot \mathbf{n} \ d\Gamma
\end{equation}
 is zeroed to ensure the impenetrability on walls, or is added to the l.h.s in the case of outflow, or, finally, $\vv_j^{n+1}$ is imposed on inflow boundaries.

Then, in the case of split elements where enrichment shape functions are used as trial functions, the resulting local system becomes:

  \begin{equation}
  \label{eqsys-poisson}
   \begin{pmatrix}
      L_{\phi,\phi} & L_{\phi,*}\\
      {L_{\phi,*}}^T & L_{*,*}
   \end{pmatrix}\;
    \begin{pmatrix}
      \delta p^{n+1}\\
      \delta {p^*}^{n+1}
   \end{pmatrix}\; = \;
   \begin{pmatrix}
      D_{\phi}\\
      D_*
   \end{pmatrix}\;
   (\hat{\vv}^{n+1})
\end{equation}

where
\begin{itemize}
 \item ${(L_{\phi,\phi})}_{N_n\times N_n} = \Delta t \displaystyle \int_{\Omega^e} \dfrac{1}{\rho} \nabla \phi_i^T \nabla \phi_j \ d\Omega$
 \item ${(L_{\phi,*})}_{N_n\times N_e} = \Delta t \displaystyle \int_{\Omega^e} \dfrac{1}{\rho} \nabla \phi_i^T \nabla \phi_j^* \ d\Omega$
 \item ${(L_{*,*})}_{N_e\times N_e} = \Delta t \displaystyle \int_{\Omega^e} \dfrac{1}{\rho} \nabla {\phi_i^*}^T \nabla \phi_j^* \ d\Omega$
 \item ${(D_{\phi})}_{N_n\times N_n} = \displaystyle \int_{\Omega^e} \nabla \phi_i^T \psi_j \ d\Omega$
 \item ${(D_*)}_{N_e\times N_n} = \displaystyle \int_{\Omega^e}  \nabla {\phi_i^*}^T \psi_j \ d\Omega$
\end{itemize}

At this point, is possible to choose between two options. First, if the enriched space expressed by equation (\ref{eq:enrich-2a}) is used, the new degrees of freedom can be included in the global system, guaranteeing continuity between elements. The second option is to follow the classical procedure of static condensation of the system through Gaussian elimination\cite{Felippa04}, where the following reduced system is obtained:
  \begin{equation}
   \left[L_{\phi,\phi} - L_{\phi,*}{(L_{*,*})}^{-1}{L_{\phi,*}}^T\right](\delta p^{n+1}) = \left[D_{\phi}- L_{\phi,*}{(L_{*,*})}^{-1}D_{*}\right](\hat{\vv}^{n+1})
   \label{condensing}
  \end{equation}
with $\delta p^{n+1} = p^{n+1}-p^{n}$, where the continuity between elements is not ensured.

% --- INICIO PARRAFO IMPORTANTE A CORREGIR

In the case of an enriched space, see equation (\ref{eq:enrich-1a}), when condensation is used, the assumption of neglecting the inter-elemental boundary terms from the integration by parts of the Poisson equation is not true. From our experience, and as reported by Coppola\cite{Coppola05}, when the Froude Number is high (that is, a large density ratio), the neglected term from the weakening of the divergence of the velocity introduces severe problems of numerical diffusion.

A way to solve this problem is to not integrate by parts the divergence term. Nevertheless, this requires imposing a pressure gradient on boundaries, whose value can not be easily predicted when there is a gravitational field, low density ratio, and the fluid is not at rest.

Finally, if condensation is used, the weakening of the velocity divergence in the Poisson step is only chosen when a large density ratio is considered. Continuous enrichment does not suffer from the above mentioned problems and the same formulation can be used for every Froude number. However, a loss of computational performance appears due to the necessity of memory management to assemble the variable-size pressure equation system.

% --- FIN PARRAFO IMPORTANTE A CORREGIR

%   If we are interested in solving the system in absolute pressure terms, the final system is
%
%   \begin{equation}
%   [L_{\phi,\phi} - L_{\phi,*}{L_{*,*}}^{-1}{L_{\phi,*}}^T](p^{n+1}) = [D_{\phi}- L_{\phi,*}{L_{*,*}}^{-1}D_*](\hat{\vv}^{n+1}) + [L_{\phi,\phi} - L_{\phi,*}{L_{*,*}}^{-1}{L_{\phi,*}^T] (p^n)
%    \label{condensing-abs}
%   \end{equation}

After obtaining the new pressure, it is necessary to correct the velocity prediction using the pressure gradient. However, in this step, the new degree of freedom for the pressure must be taken into account when calculating the enriched pressure gradient. In the case of condensed calculation, the value of $p*^{n+1}$ must be recovered from the nodal pressures calculated in the poisson step with:
\begin{equation}
  {p^*}^{n+1} = {(L_{*,*})}^{-1}[D_* \hat{\vv}^{n+1} - {L_{\phi,*}}^Tp^{n+1} + {L_{\phi,*}}^Tp^{n} + L_{*,*}{p^*}^n]
  \label{recovering}
\end{equation}

Finally, the equation system presented in Equation (\ref{correction}) must be solved.
 \begin{equation}
  \int_{\Omega} \psi \rho \vv^{n+1} d\Omega = \int_{\Omega} \psi \rho \hat{\vv}^{n+1} d\Omega - \Delta t \ \left[\int_{\Omega} \psi \nabla \delta p^{n+1} d\Omega + \int_{\Omega} \psi \nabla \delta {p^*}^{n+1} d\Omega \right]
  \label{correction}
 \end{equation}

It was previously mentioned that the pressure value is zeroed at the beginning of the time step, one of the main reasons can be found in the recovering step. It must be noted that, not only the standard nodal pressures but also the enrichments pressures of the previous iteration are required in this stage. Since the enrichment pressures depend on the interface location inside the element, using the latest $p^*$ of the previous time-step may introduce several problems due to the movement of the interface. This leads to poor results that in most cases become unstable.

Although this pressure reinitialization leads to a first order temporal approximation, successive iterations on a pressure-correction loop improve the incompressibility of the solution. Idelsohn\cite{Idelsohn13c} ensures that the stabilization effect of the first order fractional step is lost when higher order schemes are used. The same conclusion was obtained here, where instead of using a stabilization technique, a limited number of iterations (two or three) were used in order to obtain a converged pressure field without the presence of any pressure oscillations. More iterations would tend to make the scheme unstable. 