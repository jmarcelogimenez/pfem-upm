
We wish to compute the approximate solution of a dynamic flow that contains two immiscible fluids. This theory holds when we use just one fluid, but as this particular case is not the most important case of this work, we proceed considering the case with two fluids. The governing equations are the incompressible Navier-Stokes equations for both fluids, which are supplemented with the conventional boundary conditions on solid and/or open boundaries. The computational domain $\Omega$ contains both fluids, the first one (denoted by subscript 1) and the other (with its corresponding variables denoted by the subscript 2) of densities and viscosities $\rho_i$ and $\mu_i$ $(i=1,2)$, respectively, being $\Gamma$ the boundary of $\Omega$. The governing equations written in a Lagrangian framework are:

\begin{eqnarray}
% \nonumber to remove numbering (before each equation)
  \nabla \cdot \mathbf{v} &=& 0 \\
  \rho\frac{D\mathbf{v}}{Dt} &=& -\nabla p + \rho \mathbf{f}+ \mu \nabla^2 \mathbf{v}
\end{eqnarray}

where the convective term does not appear in this Lagrangian formulation but a kinematic problem will be solved at each time step. Here $v$, $p$ are the velocity and fluid pressure and $\mathbf{f}$ are the external body forces (normally gravity and/or inertial forces). Apart from this, each particle carries the information of the fluid that was initially assigned, this quantity $\lambda$ has values -1 or 1 depending if it belongs to the first or second fluid. This function is projected to the mesh nodes to determine the free surface position, which is defined as the set of point with $\lambda=0$

Let us explain the algorithm that PFEM-2 follows in order to compute a complete time step. Let us assume that all fluid variables are known at time $t_n$ for the particles and the mesh nodes representing the fluids. Subindexes $()_j$ y $()_p$ represent a generic node $j$ and a generic particle $p$ respectively. Let $\phi$ and $\psi$ the pressure and velocity finite element basis functions. According to this notation

\begin{enumerate}
  \item Acceleration Stage: Calculate acceleration components: $\mathbf{a}_{\tau}$ (viscous component) and $\mathbf{a}_{p}$ (pressure component) on the mesh nodes.


  \begin{equation}\label{Step1a}
\int_{\Omega}\mathbf{a}^{n}_{\tau}\psi_j d\Omega=\int_{\Omega}\mu \nabla^{2}\mathbf{v}^{n} \psi_j d\Omega=-\int_{\Omega}\mu \nabla\mathbf{v}^{n} \nabla \psi_j d\Omega + \int_{\Gamma}\mu \nabla\mathbf{v}^{n} \psi_j d\Omega
\end{equation}

\begin{equation}\label{Step1b}
\int_{\Omega}\mathbf{a}^{n}_{p}\psi_j d\Omega=-\int_{\Omega}\nabla p^{n} \psi_j d\Omega=\int_{\Omega} p^{n} \nabla \psi_j d\Omega - \int_{\Gamma} p^{n} \psi_j d\Omega
\end{equation}

\begin{equation}\label{Step1c}
\mathbf{a}^{n}=\mathbf{a}^{n}_{p} + (1-\theta)\mathbf{a}^{n}_{\tau}
\end{equation}

In case we consider two fluids we use: $\theta=1$ y $p^n=0$.

  \item X-IVAS Stage: Evaluate new particle position and intermediate velocity following the velocity streamlines

  \begin{equation}\label{Step2a}
\mathbf{x}^{n+1}_{p}=\mathbf{x}^{n}_{p} + \int_{t_n}^{t_{n+1}} \mathbf{v}^{n}(\mathbf{x}(\alpha)_{p}) \ d\alpha
\end{equation}

\begin{equation}\label{Step2b}
\widehat{\widehat{\mathbf{v}}}^{n+1}_{p}=\mathbf{v}^{n}_{p} + \int_{t_n}^{t_{n+1}} \left(\mathbf{a}^{n}(\mathbf{x}_{p}) + \mathbf{f}^{\alpha} (\mathbf{x}(\alpha)_{p})\right)  \ d\alpha
\end{equation}

Here some comments are required. As the integration is along the streamlines, the time step $\Delta t=t^{n+1}-t^{n}$ is divided into $N$ substeps where the velocity field is frozen for the mesh nodes and particles evolve from one sub-position to the next driven by the interpolated velocity field at the new sub-position. Temporal integration for the position and velocity can be solved using analytical expressions\cite{} or high-order integrators\cite{}. However, in this work a sub-stepping integrator inherited from STS\cite{} is used, which can adapt its sub-step $\delta t$ depending on the local CFL number. A more exhaustive explanation of the X-IVAS can be found in \cite{Gimenez}. According to this sub-stepping integration the equations (\ref{Step2a}) and (\ref{Step2b}) can be written as:

\begin{equation}\label{Step2astep}
\mathbf{x}^{n+1}_{p}=\mathbf{x}^{n}_{p} + \sum_{i=1}^{N} \mathbf{v}^{n}(\mathbf{x}^{n+\frac{i}{N}}_{p}) \delta t
\end{equation}

\begin{equation}\label{Step2bstep}
\widehat{\widehat{\mathbf{v}}}^{n+1}_{p}=\mathbf{v}^{n}_{p} + \sum_{i=1}^{N} \left(\mathbf{a}^{n}(\mathbf{x}^{n+\frac{i}{N}}_{p}) + \mathbf{f}^{n} (\mathbf{x}^{n+\frac{i}{N}}_{p})\right)  \delta t
\end{equation}

  \item Projection Stage: Project velocity and level set function from the particles to the mesh:
  \begin{equation}\label{Step3a}
\displaystyle \widehat{\widehat{\mathbf{v}}}^{n+1}_{j}=\frac{\sum_{p} \mathbf{v}^{n+1}_{p} W(\mathbf{x}_{j}-\mathbf{x}_{p})}{\sum_{p} W(\mathbf{x}_{j}-\mathbf{x}_{p})}
\end{equation}

\begin{equation}\label{Step3b}
\displaystyle \lambda^{n+1}_{j}=\frac{\sum_{p} \lambda^{n+1}_{p} W(\mathbf{x}_{j}-\mathbf{x}_{p})}{\sum_{p} W(\mathbf{x}_{j}-\mathbf{x}_{p})}
\end{equation}

Where the functions $W$ are the typical kernel functions used in particle methods as for example SPH \cite{Monaghan} and summations are extended to the particles within a critical distance that depends on the election of the kernel function. 

  \item Implicit Viscosity Stage: Implicit correction of the viscous diffusion with FEM:

 \begin{eqnarray}\label{Step4a}
\displaystyle \widehat{\mathbf{v}}^{n+1}_{j}=\widehat{\widehat{\mathbf{v}}}^{n+1}_{j} + \theta \Delta t \mathbf{a}^{n+1}_{\tau}
\end{eqnarray}

In this part the velocity boundary conditions are imposed. 

 \item Poisson Stage: Search the pressure value solving the Poisson equation system with FEM:

 
 \begin{eqnarray}\label{Step5a}
 % \nonumber to remove numbering (before each equation)
   \ \nabla \cdot [\frac{\Delta t}{\rho}\nabla(\delta p^{n+1})] &=& \nabla \cdot \widehat{\mathbf{v}}_j^{n+1} \\
   \frac{\partial p}{\partial n} &=& 0 \quad \Gamma_D \\
   p &=& 0 \quad in \Gamma_N \\
 \end{eqnarray}
  

 \item Correction Stage: Update the mesh and particle velocity with pressure and diffusion corrections:
 \begin{equation}\label{Step6a}
  \int_{\Omega} \rho_j \mathbf{v}_j^{n+1}\ d\Omega \ = \ \int_{\Omega} \rho_j  \widehat{\mathbf{v}}_j^{n+1}\ d\Omega\ - \int_{\Omega} \Delta t \ (\mathbf{a}_j^{n+1} - \mathbf{a}_j^{n})\ d\Omega
 \end{equation}
  \begin{equation}\label{Step6b}
  \rho_p \mathbf{v}_p^{n+1}\  = \ \rho_p \widehat{\widehat{\mathbf{v}}}_p^{n+1} + \sum_{j} \delta \mathbf{v}_j^{n+1} \psi_j(\mathbf{x}_{p})
  \end{equation}
  where $\delta \mathbf{v}_j^{n+1} = \mathbf{v}_j^{n+1}-\widehat{\widehat{\mathbf{v}}}_j^{n+1}$.

%  \item 
%  \item 
%  \item 
\end{enumerate} 