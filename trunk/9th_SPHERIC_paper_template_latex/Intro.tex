
Particle methods have started to combine the Lagrangian part with a grid that fixed or re-meshed every k time steps supports part of the pressure and velocity calculation. The original idea given by Monaghan \cite{Mon77} or later works applied to fluid mechanics \cite{Monaghan88} where a pure Lagrangian perspective was used during the whole meshless computation has been in some cases completed using other well known discretization methods as FVM \cite{Quinlan} or FEM \cite{Calvo}. We could try to find the first combination of Lagrangian methods and FEM methods in the paper \cite{Ide03b} where a extended Delaunay Tesellation is used to reconstruct the mesh while the fluid evolves. In this method known as MFEM, the construction of the shape functions inside each polyhedron is based on a non-Sibsonian interpolation.

The next step in this evolution was the first version of the PFEM method \cite{Idelsohn04}, which was a robust method designed to solve fluid–structure interaction problems including free surface, breaking waves, flow separations, etc...where lagrangian particles and meshing processes are alternated with the advantage of having a FEM structure that supports the differential equation solvers. An interesting difference between the PFEM and other particle methods as SPH or MPM \cite{Wieckowsky04} is that the mesh particles do not transport mass and consequently have a volume, in contrast PFEM uses non material points that transport fluid properties with fixed density.

Other methods that also combine both Eulerian and Lagrangian perspectives are the arbitrary Eulerian-Lagrangian (ALE)\cite{Donea83} methods or the semi-Lagrangian methods \cite{Bermejo}. The Lagrangian perspective makes possible to use a material derivative formulation where the absence of the non linear convective terms transform the Navier-Stokes system into a transformed linear coupled problem. Methodologies like the backward Characteristics method \cite{Bermejo} gives also this possibility but if the process is done in a fixed mesh without any distortion, a diffusion process appears due to the interpolation of the feet of characteristics.

Most of the methods cited before including PFEM have a uncomfortable drawback which is the necessity of constructing or controlling the mesh quality during the simulation if an accuracy of the solution has to be maintained. The evaluation of the mesh distortions or the re-meshing processes are always computationally expensive and it would be interesting exploring the possibility of avoiding that step. Consequently a new generation of the PFEM methodology could be developed in such a way that no re-meshing is necessary.
In contrast to the backward characteristic method where the feet of the characteristic line was searched and located in a mesh element based on the known velocity fields at past time steps, a new strategy known as X-IVAS (eXplicit Integration following the Velocity and Acceleration Streamlines)\cite{Idelsohn12}. This methodology uses the streamlines at the present time step instead of the the particle trajectories to convect the fluid particles. The use of this strategy on a fixed mesh and interpolate the particle information to the mesh nodes gives a new improved method known as PFEM-2\cite{Idelsohn12b}. The use of particles give the possibility of solving complex accurate flows with large time steps, while the fixed mesh allows accurate solutions of the fractional step method without any remeshing process. Mesh nodes and moving particles interchange information using interpolation processes using different strategies. A detailed explanation of the PFEM-2 method is given in Section \ref{PFEM_Algorithm}.

In this work the PFEM-2 has been used to solve free surface flows in different problems starting from classical benchmark problems as the lid driven cavity and finishing with problems of industrial interest. As in most of the FEM derived methodologies the introduction of a complex geometry is not a problem. The treatment of the free surface has been done simulating both fluids that share the interface and using a scalar function to identify each fluid. To improve the pressure calculation along the free surface different versions of the enrichment technique \cite{Coppola} has been used. The computation of the free surface at different positions has been performed and compared to other well reputed and commercial Eulerian codes with good results but using larger time steps with any loss of numerical stability. 