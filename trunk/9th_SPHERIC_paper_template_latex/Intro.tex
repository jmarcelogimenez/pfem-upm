
Every day, hybrid methods gain traction in computational fluid mechanics, where the Lagrangian framework given by a particle method, is combined with a Eulerian methodology. In these hybrid methods, a fixed or reconstructed grid supports part of the pressure and velocity calculation. The original idea, proposed by Monaghan \cite{Mon77} and later works applied to fluid mechanics \cite{Monaghan88}, where a pure Lagrangian perspective was used during the whole meshless computation, has been in some cases completed using other well known discretization methods, such as FVM \cite{Nestor20091733} or FEM \cite{Ide03}. The first combination of Lagrangian and FEM methods can be found in \cite{Ide03b}, where an extended Delaunay Tesellation is used to reconstruct the mesh while the fluid evolves. In this method, known as MFEM, the construction of the shape functions inside each polyhedron is based on a non-Sibsonian interpolation.

%PROBLEMAS PIC en : http://www.flow3d.com/home/resources/cfd-101/general-cfd/lagrangian-particles
The next step in this evolution was the first version of the PFEM method \cite{Idelsohn04}, which was a robust method designed to solve fluid-structure interaction problems including free-surface, breaking waves, flow separations, etc... where lagrangian particles and meshing processes are alternated with the advantage of having a FEM structure that supports the differential equation solvers. An interesting difference between the PFEM and other hybrid methods as PIC (particle-in-cell) \cite{Harlow55}, MAC (marker-and-cell) \cite{Harlow65} and MPM (material point method) \cite{Wieckowsky04} is that while in latter methods the particles transport mass and consequently have a volume, PFEM particles are non-material points that transport the fixed intensive properties of the fluid.

Other methods, that also combine both Eulerian and Lagrangian perspectives, are the arbitrary Eulerian-Lagrangian (ALE) \cite{Donea83} or semi-Lagrangian methods \cite{Bermejo}. The Lagrangian perspective makes it possible to use a material derivative formulation where the absence of the non-linear convective terms transform the Navier-Stokes system into a transformed linear coupled problem. Methodologies, such as the backward Characteristics method \cite{Bermejo}, also give this possibility but if the process is done in a fixed mesh without any distortion, and unless high order polynomials are used, a dissipative process appears due to the interpolation of the feet of characteristics.

In contrast to the backward characteristic method where the feet of the characteristic line was searched and located in a mesh element based on the known velocity fields at past time steps, a new strategy known as X-IVAS (eXplicit Integration following the Velocity and Acceleration Streamlines) was developed by Idelsohn et al.\cite{Idelsohn12}. This methodology of integrating the convection of fluid particles is based on following the streamlines of the flow in the current time step instead of the particle trajectories, which represents an alternative way to solve the non-linearities of the flow equations. Adding this strategy to the original PFEM method, a new methodology appears called Particle Finite Element Method Second Generation (PFEM-2) \cite{Idelsohn12b}. The X-IVAS strategy gives the possibility of solving complex flows with large time steps ($CFL>1$), as well as the presence of the mesh allows for accurate solutions of the fractional step method.

In PFEM-2, there are two approaches to communicate particle and mesh data, each one generating two versions of the method. The first one is called \textit{Moving Mesh}, which follows the original idea of PFEM, creating a new mesh using the new position of the particles as nodes. The second version, named \textit{Fixed Mesh}, projects the particle states to nodes while preserving the initial background mesh. The former strategy maintains the uncomfortable drawback of previously cited methods, which leads to the necessity of constructing or controlling the mesh quality during the simulation if the accuracy of the solution has to be maintained. The evaluation of the mesh distortions or the re-meshing processes are always computationally expensive and it would be interesting to explore the possibility of avoiding that step. Consequently, the \textit{Fixed Mesh} approach avoids the remeshing at each time-step. In this approach, mesh nodes and moving particles interchange information through different interpolation algorithms. In the context of this paper, PFEM-2 will refer to the \textit{Fixed Mesh} version, and a detailed explanation of this algorithm is given in Section \ref{PFEM_Algorithm}.

% Most of the methods cited before including PFEM have a uncomfortable drawback which is the necessity of constructing or controlling the mesh quality during the simulation if an accuracy of the solution has to be maintained. The evaluation of the mesh distortions or the re-meshing processes are always computationally expensive and it would be interesting exploring the possibility of avoiding that step. Consequently a new generation of the PFEM methodology could be developed in such a way that no re-meshing is necessary.
% In contrast to the backward characteristic method where the feet of the characteristic line was searched and located in a mesh element based on the known velocity fields at past time steps, a new strategy known as X-IVAS (eXplicit Integration following the Velocity and Acceleration Streamlines) was developed\cite{Idelsohn12}. This methodology uses the streamlines at the present time step instead of the the particle trajectories to convect the fluid particles. The use of this strategy on a fixed mesh and interpolate the particle information to the mesh nodes gives a new improved method known as PFEM-2 \cite{Idelsohn12b}. The use of particles give the possibility of solving complex accurate flows with large time steps ($CFL>1$), while the fixed mesh allows accurate solutions of the fractional step method without any re-meshing process. Mesh nodes and moving particles interchange information using interpolation processes using different strategies. A detailed explanation of the PFEM-2 method is given in Section
% \ref{PFEM_Algorithm}.

The extension of PFEM-2 to multi-fluids flows is presented by Idelsohn et al. \cite{Idelsohn13c} where novel features for, mainly, the treatment of the interface evolution are explained. In that work, the treatment of the free-surface has been done simulating both fluids that share the interface using a scalar function to identify each fluid, and, to improve the pressure calculation close to the free surface, a discontinuous enrichment technique is employed. However, the reference has a lack of validation because only non-viscous fluids with high density ratios were analyzed, skipping the free surface numerical problems derived when viscosity plays a dominant role and/or the density ratio is moderate.

In the current work, the mentioned PFEM-2 for multifluids strategy has been used to solve a wider range of free-surface flows, starting from classical benchmark problems, such as the Rayleigh-Taylor instability (Section \ref{sec:rt}), and finishing with problems of industrial interest, such as sloshing in closed cointainers (Section \ref{sec:Ansari}) or dam-break cases (Section \ref{sec:db}), where the obtained results must be validated with numerical or experimental data. However, to solve all the range of Froude number situations, a novel continuous version of the enrichment methodology is presented. It can be used on general cases in order to avoid the problems encountered particularly in high Froude number flows where other versions present poor accuracy. The obtained results have been also compared to other well reputed Eulerian codes, obtaining accurate and numerically stable results while using larger time steps. Taking into account that the computational time per time-step is comparable to the one required by the other methodologies, the possibility of increasing the time-step implies shorter global computational times.
