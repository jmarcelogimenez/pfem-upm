\subsection{Particular formulation for two fluids.}
The following simplifications of the general algorithm described above are considered:

\begin{itemize}
  \item $\theta=1$
  \item $p^n=0$
\end{itemize}

As the initial pressure $p^n$ and the factor $1-\theta$ are both zero, the first stage of the general formulation can be supressed and the algorithm starts with the X-IVAS step.

\begin{enumerate}

\item X-IVAS Stage: Evaluate new particle position and intermediate velocity following the velocity streamlines

 \begin{equation}\label{Step2astep2fluids}
\mathbf{x}^{n+1}_{p}=\mathbf{x}^{n}_{p} + \sum_{i=1}^{N} \mathbf{v}^{n}(\mathbf{x}^{n+\frac{i}{N}}_{p}) \delta t
\end{equation}

\begin{equation}\label{Step2bstep2fluids}
\widehat{\widehat{\mathbf{v}}}^{n+1}_{p}=\mathbf{v}^{n}_{p} + \sum_{i=1}^{N} \mathbf{f}^{n} (\mathbf{x}^{n+\frac{i}{N}}_{p})  \delta t
\end{equation}

\item Projection Stage: Project velocity similarly to equation (\ref{Step3a}) and the scalar function $\lambda$ from the particles onto the mesh nodes:

\begin{equation}\label{Step3b}
\displaystyle \lambda^{n+1}_{j}=\frac{\sum_{p} \lambda^{n+1}_{p} W(\mathbf{x}_{j}-\mathbf{x}_{p})}{\sum_{p} W(\mathbf{x}_{j}-\mathbf{x}_{p})}
\end{equation}

\item Implicit Viscosity Stage: Implicit correction of the viscous diffusion, see equation (\ref{Step4a}).

\item Poisson Stage: Pressure correction $\delta p^{n+1}$ computation on the mesh nodes by solving the Poisson equation system. Equation (\ref{Step5a}) is solved considering that density could change its value inside those elements where the $\lambda$ function changes its sign. For such elements, crossed by the free surface, will be subdivided into sub-elements of constant density according to the free surface intersections, see section \ref{GeneralFor}.

\item Correction Stage: Update the mesh and particle velocity with pressure and diffusion corrections similarly to equations (\ref{Step6a}) and (\ref{Step6b}) but taking into account the density change in those elements close to the free surface.

    Due to the discontinuities in the density and viscosity fields, and consequently the computation of the pressure gradients near the interface between both fluids, the last two stages (Poisson and Correction) in this process must be performed using a enrichment technique for the pressure field that deserves a detailed explanation in the next section.

\end{enumerate}
